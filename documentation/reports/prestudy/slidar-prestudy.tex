\documentclass{article}
\usepackage[utf8]{inputenc}
\usepackage{graphicx}
\title{\includegraphics[width=0.5\textwidth]{UU_logo.pdf}\\
Construction of the Slidarr}
\author{Mats Jonsson, Sören Meinken, Mohammad El Musleh}
\date{March 2019}

\begin{document}

\maketitle

\section{Introduction}
The idea is to create a new revolutionary musical instrument called Slidarr. It originated from using the guitar instrument but instread of pulling the strings, the fingers will slide on two of the strings without the strings making any sound. The sound will come from the relative location of the fingers on the strings which will be determined by the electronic equipment developed in this project.

The Slidarr senses tiny changes in resistance and translates it to MIDI signals before sending it over USB to a computer running an software synth that produces the audio. The instrument's interface will consist of a metal wire with a constant current running through it. The artist touches the wire at any position with a conductor, that reads a tiny voltage drop depending on its position along the wire. The voltage drop is amplified and fed into an ADC converter on the ARM Cortex, that translates the signal to a corresponding MIDI note.

\section{Objectives}
On a guitar there are usually six strings, for this prototype the minimum objective is to get two strings running first. In addition to that, the project will be started using optimal conditions for the strings and the conductor on the hand. This means using a very good conductor on the finger to bridge the strings, so that the biggest resistance is the length of the wires. If this works, the next move would be to use bare hands with possibly other strings that have a similar resistance. Once the MIDI data is created from the ADC there are several options what to do with it e.g. send it to a computer/syntheziser via USB, wirelessly or synthezise the signal on chip.

\section{System overview}


There are five major parts in this project to be realized:
\begin{itemize}
    \item Creating/developing a measurement circuit to read the resistance(distance) from the strings
    \item Read that signal with an ADC
    \item Convert the signal to a proper representation for the MIDI protocol
    \item Send the signal to a syntheziser
    \item Calibration and setting of modes
\end{itemize}

Hardware needed:
\begin{itemize}
    \item strings
    \item circuit
    \item microcontroller with ADC
    \item calibration buttons, leds for mode?
    \item USB or wifi tranceiver (or only transmitter?)
\end{itemize}

--- drawing ---


\section{Organisation}
The five major parts mentioned before can all be developed on in parallel. There will be three people working available to work in this project.

\section{Method}

\section{Challenge}
To get a proper measurement so that notes can be created from it and beautiful music is created.


\end{document}

